
	\section{Règles du jeu}
	\label{sec:regles}

	\subsection{But du jeu}
	\label{subsec:butdujeu}
	BedBihan est un jeu de stratégie au tour-par-tour où chaque joueur dirige un peuple sur une carte. Le but est de marquer le plus de points possible en dirigeant les unités de son peuple pour qu'elles occupent au mieux les cases de la carte. Les unités d’un joueur peuvent attaquer les unités d’un autre joueur pour les détruire, limitant ainsi l'acquisition de points adverses. La partie se termine quand le nombre de tours imparti est atteint, ou qu'il ne reste qu'un seul joueur en jeu --- les autres ayant perdu toutes leurs unités\footnote{Pour ce projet, le nombre de joueurs sera limité à 2.}.

	\subsection{Carte du monde}
	\label{subsec:carte}
	La carte du monde se compose de cases hexagonales. Chaque case possède un type de terrain : Plaine,	Désert, Montagne ou Forêt. Chaque case peut être occupée par une ou plusieurs unités alliées, mais deux unités adverses ne peuvent se trouver sur la même case. Par défaut, une case occupée par une ou plusieurs unités alliées rapporte 1 point. Des bonus et malus sont possibles selon les caractéristiques de chaque peuple (voir Sous-section \ref{sec:peuples}). 
	
	À chaque nouvelle partie, la carte est générée aléatoirement avec le même nombre de case de type Plaine, Désert, Montagne et Forêt, afin de ne pas avantager un peuple. 
	La carte peut avoir trois tailles différentes :
	\begin{itemize}
		\item Démo : 4x4 cases, 5 tours, 4 unités par peuple.
		\item Petite : 10x10 cases, 20 tours, 6 unités par peuple.
		\item Normale : 14x14 cases, 30 tours, 8 unités par peuple.
	\end{itemize}
	
	
	\subsection{Peuples}
	\label{subsec:peuples}
	BedBihan permet de jouer avec trois peuples différents : les Elfes, les Humains et les Korrigans. Chaque peuple présente des caractéristiques particulières\footnote{Ces caractéristiques se limitent pour l'instant à celles spécifiées dans les consignes du projet. Elles sont toutefois susceptibles d'être étoffées en cours de développement.}.
	
		Pour les Elfes :
		\begin{itemize}
			\item le coût de déplacement sur une case Forêt est divisé par deux. 
			\item le coût de déplacement sur une case Désert est multiplié par deux.
			\item Une unité elfe a 50\% de chance de survivre avec 1 point de vie lors d’un combat conduisant normalement à sa mort.
		\end{itemize}
		
		Pour les Humains :
		\begin{itemize}
			\item le coût de déplacement sur une case Plaine est divisé par deux.
			\item une unité humaine ne rapporte aucun point lorsqu'elle occupe une case Forêt.
			\item lorsqu’une unité humaine détruit une autre unité, elle marque 1 point. Les points ainsi gagnés sont liés à l'unité : si celle-ci meurt, les points sont perdus.
		\end{itemize}
		
		Pour les Korrigans :
		\begin{itemize}
			\item le coût de déplacement sur une case Plaine est divisé par deux.
			\item une unité korrigane ne rapporte aucun point lorsqu'elle occupe une case Plaine.
			\item lorsqu'une unité korrigane occupe une case Montagne, elle peut se déplacer sur n’importe quelle autre case Montage de la carte, à condition que celle-ci ne soit pas occupée par une unité adverse.
		\end{itemize}

	\subsection{Unités}
	Chaque unité possède des points de vie, des points de déplacement, des points d'attaque et des points de défense. Les points de vie représentent la santé de l'unité : plus ceux-ci sont élevés, plus l'unité peut attaquer ou se défendre efficacement (voir Sous-section \ref{sec:combats}). Lorsque les points de vie d'une unité atteint 0, cette dernière est détruite. Ces points de vie ne se régénèrent pas d'un tour à l'autre.
	
	Les points d'attaque représentent la force brute d'une unité lorsqu'elle en attaque une autre.  Plus ceux-ci sont élevés, plus l'unité infligera de dégâts à l'adversaire. De la même façon, les points de défense représentent la résistance d'une unité face à une attaque : plus ceux-ci sont élevés, et moins l'unité perdra de points de vie dans un combat. 
	
	Les points de déplacement, quant à eux, représentent la capacité de mouvement d'une unité à travers la carte. Plus ces points sont élevés, plus l'unité pourra se déplacer loin lors d'un tour (en tenant compte des bonus et malus dûs aux types de terrain, propres à chaque peuple). Ces points de déplacement sont restaurés d'un tour à l'autre.
	
	\subsection{Déroulement d'un tour}
	Lorsque vient le tour d'un joueur, celui-ci peut bouger chacune de ses unités tant qu'elles ont suffisamment de points de déplacement. Par défaut, se déplacer d'une case coûte 1 point de déplacement. Une unité peut également passer son tour.
	
	Lorsqu'une unité tente de se déplacer sur une case sur laquelle se trouve une ou plusieurs unités ennemies, un certain nombre de combat s'engagent (voir Sous-section \ref{sec:combats}). Une fois les combats terminés, si toutes les unités adverses ont été détruites, l'unité se déplace effectivement sur la case, en consommant le nombre de points de déplacement nécessaire. Si au moins une unité ennemie est encore en vie, l'unité alliée reste sur sa case actuelle mais consomme malgré tout le même nombre de points de déplacement.
	
	Ainsi, une unité peut attaquer et se déplacer autant de fois que le permet son nombre de points de déplacement. Lorsqu'un joueur a fini de déplacer toutes ses unités, c'est au joueur suivant de commencer son tour, et ainsi de suite.

	\subsection{Combats}
	\label{subsec:combats}
	Un combat a lieu lorsqu'une unité tente de se déplacer sur une case occupée par l'ennemi. Une unité attaque seule et ne peut attaquer qu'une seule unité adverse à la fois : dans le cas où plusieurs unités ennemies se situent sur la même case, c'est l'unité adverse ayant la défense la plus élevée qui est affrontée.
	
	Un combat se compose d’un certain nombre d’attaques. Ce	nombre est choisi aléatoirement à l’engagement : il varie entre 3 et le nombre de points de vie de l’unité avec le plus de points de vie + 2. Le combat s’arrête lorsque ce nombre est atteint, ou que l'une des unité est détruite. 
	
	À chaque attaque, un calcul de probabilités de perte de points de vie est effectué. Par exemple, si l’unité attaquante a 3 en attaque et l’unité défensive 4 en défense, l’attaquant à 75\% de chance de perde des points de vie --- le rapport de force étant de 3/4. Inversement, l'attaqué n'a que 25\% de chance de perdre des points de vie. Au contraire, si l'attaquant possède 4 en attaque contre 1 en défense, l'attaquant n'a que Les points de vie entrent en compte dans le calcul des probabilités : si une unité attaquante ayant	4 en attaque possède 50\% de sa vie, alors son attaque sera au final de 4*50\% = 2. L’unité attaquée suit le même calcul pour sa défense.

	À la fin d’un combat gagné par l’attaquant et si la case du vaincu ne contient plus d’unité, l’attaquant se déplace automatiquement sur cette case. Cela coûte alors le coût d’un déplacement (bonus et malus inclus). Le même calcul s’applique lorsque l’unité attaquante ne détruit pas l’unité adverse.	Cela signifie qu’une unité qui vient d’attaquer peut très bien se déplacer ensuite si elle possède les points de déplacement nécessaires. Lorsqu’un joueur n’a plus d’unité, il est éliminé. Lorsqu’il ne reste plus qu’un seul joueur dans une partie, celui-ci a gagner. Une unité ne regagne pas ses points de vie d’un tour à un autre.
	
	
	

	
	
