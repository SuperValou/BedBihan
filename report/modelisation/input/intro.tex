\section{Introduction}
	
	Ce rapport s'inscrit dans le projet \emph{Modélisation et Programmation Orientées Objet}, dans le cadre de la formation de quatrième année informatique de l'INSA de Rennes. L'objectif de ce projet et la réalisation d'un jeu librement inspirée du jeu de plateau éponyme crée par Philippe Keyaerts en 2009. Le jeu ainsi réalisé portera le nom de \emph{Bedbihan}, traduction bretonne du nom du jeu original. 

	\emph{Bedbihan} est un jeu à deux joueurs, controlées par deux utilisateur par l'intermediare d'un même ordinateur. Il s'agit d'un jeu au tour par tour où chaque joueur choisi une race dont il controlera un certain nombre d'unité sur un plateau. Le but du jeu est d'arrivé au bout d'un certain nombre de tour avec le maximum de point, déterminé par la position de ces unités sur le plateau de jeu. 

	Ce rapport présente les choix principaux déterminés lors de la phase d'analyse et de conception. Dans un premier temps, nous présenterons les fonctionalités principales du jeu à l'aide de cas d'utilisations. Dans un second temps, des diagrames d'états-transitions nous permettrons de présenter les fonctionnement des différents objects. Ce qui nous permettra dans un troisiéme temps de présenter les différents choix pris a l'aide du diagramme de classe et le détail des patrons de conceptions utilisés. 





















