\section*{Introduction}\addcontentsline{toc}{section}{Introduction}
	
	Ce document est le premier rapport du projet \emph{Modélisation et Programmation Orientées Objet}, qui a lieu dans le cadre de la formation de quatrième année informatique de l'{\sc Insa} de Rennes. L'objectif de ce projet est la réalisation d'un jeu vidéo librement inspirée de \emph{Small World}, un jeu de plateau créé par Philippe Keyaerts en 2009. Le jeu ainsi réalisé portera le nom de \emph{Bedbihan}, traduction bretonne du nom du jeu original. 

	La première partie de ce rapport présentera les règles précises du jeu, déterminées à partir de celles formulées dans le cahier des charges initial. La deuxième partie illustrera à l'aide de diagrammes les différentes phases de jeu, les actions possibles pour le joueur, les interactions entre les différents éléments ainsi que leur séquencage dans le temps. Enfin, la troisième partie présentera la modélisation de notre programme en précisant les différents patrons de conceptions utilisés.
	
	Pour des raisons de cohérence avec le monde professionnel dans le développement de notre projet, \emph{Bedbihan} sera intégralement en anglais. Par conséquent, bien que le texte de ce rapport soit en français, l'ensemble de ses illustrations, figures et diagrammes, sont en anglais.






	












