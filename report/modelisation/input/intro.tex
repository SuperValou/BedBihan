\section{Introduction}
	
	Ce document est le premier rapport du projet \emph{Modélisation et Programmation Orientées Objet}, qui se déroule le cadre de la formation de quatrième année informatique de l'INSA de Rennes. L'objectif de ce projet et la réalisation d'un jeu librement inspirée du jeu de plateau éponyme crée par Philippe Keyaerts en 2009. Le jeu ainsi réalisé portera le nom de \emph{Bedbihan}, traduction bretonne du nom du jeu original. 

	\emph{Bedbihan} est un jeu à deux joueurs, controlées par deux utilisateurs par l'intermediare d'un même ordinateur. Il s'agit d'un jeu au tour par tour où chaque joueur choisi une race dont il controlera un certain nombre d'unité sur un plateau. Le but du jeu est d'arrivé au bout d'un certain nombre de tour avec le maximum de point, déterminé par la position de ces unités sur le plateau de jeu. 

	Ce rapport présente les choix principaux déterminés lors de la phase d'analyse et de conception. Dans un premier temps,la phase d'analyse présentera :
	\begin{itemize}
		\item les différentes phases du jeu à l'aide de diagramme d'état-transition.
		\item les différentes interactions possibles pour l'utilisateurs à l'aide de diagramme de cas d'utilisations.
		\item les interactions entre les différents objets du jeu et leur séquencage dans le temps grace au diagramme de séquence.  
	\end{itemize}

	Des points plus précis tels que la gestion des combats et le déroulement d'un tour seront évoqué. 

	Cette première phase d'analyse nous permettra de présenter dans un second temps la modélisation précise de notre programme grace au diagramme de classe. Nous préciserons l'emploi des différents patrons de conceptions utilisés.



















