\section{Introduction}
	
	Ce document est le premier rapport du projet \emph{Modélisation et Programmation Orientées Objet}, qui a lieu dans le cadre de la formation de quatrième année informatique de l'INSA de Rennes. L'objectif de ce projet et la réalisation d'un jeu vidéo librement inspirée du jeu de plateau du même nom créé par Philippe Keyaerts en 2009. Le jeu ainsi réalisé portera le nom de \emph{Bedbihan}, traduction bretonne du nom du jeu original. 

	\emph{Bedbihan} est un jeu à deux joueurs qui se joue par l'intermediare d'un même ordinateur. Il s'agit d'un jeu au tour par tour où chaque joueur choisi une race dont il controlera un certain nombre d'unités sur un plateau. Le but du jeu est d'arrivé au bout d'un certain nombre de tour avec le maximum de points, déterminés par la position de ces unités sur le plateau de jeu. 

	Ce rapport présente les choix principaux fixés lors de la phase d'analyse et de conception. Dans un premier temps, la phase d'analyse reprendra les régles du jeux tel que formulés dans le cachier des charges initials avec quelques modifications.
	Puis dans un deuxiéme temps, la phase d'analyse présentera :
	\begin{itemize}
		\item les différentes phases du jeu à l'aide d'un diagramme d'état-transition.
		\item les différentes interactions possibles pour l'utilisateurs à l'aide de diagrammes de cas d'utilisations.
		\item les interactions entre les différents objets du jeu et leur séquencage dans le temps grace aux diagrammes de séquences.  
	\end{itemize}
	Des points plus précis tels que la gestion des combats et le déroulement d'un tour seront évoqué. 

	Enfin, nous seront à même de présenter dans un troisième temps la modélisation de notre programme grace au diagramme de classe. Nous préciserons l'emploi des différents patrons de conceptions utilisés.



















