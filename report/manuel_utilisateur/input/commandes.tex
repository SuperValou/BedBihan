\newpage
\section{Commandes}

		\subsection{Lancement du jeu}

		Au lancement de \emph{Bedbihan}, l'utilisateur peut entrer les paramètres d'une nouvelle partie . Le lancement d'une nouvelle partie implique d'entrer de le nom des deux joueurs, choisir une taille de carte et de sélectionner un peuple pour chacun des joueurs.
		
		\begin{figure}[h!]
			\begin{center}
				\includegraphics[width=0.8\textwidth]{figure/launch.jpg}
			\end{center}
			\caption{Menu principal, permettant de créer une nouvelle partie.}
			\label{fig:launch}
		\end{figure}

		\subsection{Sélection d'une unité}
		Chaque joueur, lorsque c'est son tour, peut cliquer sur les cases de la carte (en utilisant le {\bf clic-gauche de la souris}) pour connaitre les unités actuellement situées sur cette case. Les unités s'affichent à gauche, et par défaut la première unité de la liste est sélectionnée. Le joueur peut cliquer sur une autre unité de cette liste pour la sélectionner (en utilisant le {\bf clic-gauche de la souris}). Dans le panneau latéral de droite, les statistiques de l'unité sélectionnées s'affichent : points de vie, attaque, défense, etc. sont indiquées en détails.
		
		\begin{figure}[h!]
			\begin{center}
				%\includegraphics[width=0.8\textwidth]{figure/unit_selected.jpg}
			\end{center}
			\caption{À gauche, la liste des unités présentes sur la case. À droite, les statistiques de l'unité sélectionnée.}
			\label{fig:launch}
		\end{figure}
		
		\subsection{Déplacement d'une unité}
		Si l'unité sélectionnées est une unité appartenant au joueur, les cases sur lesquelles il peut la déplacer voient leur contour s'afficher en vert. Pour bouger une unité, le joueur doit cliquer sur une case dont le coutour est en vert (en utilisant le {\bf clic-droit de la souris}).
		
		\begin{figure}[h!]
			\begin{center}
				%\includegraphics[width=0.8\textwidth]{figure/unit_selected.jpg}
			\end{center}
			\caption{Cases sur lesquelles l'unité sélectionnée peut se déplacer.}
			\label{fig:launch}
		\end{figure}
		
		\paragraph{Remarque} Le clic-gauche sert à sélectionner une case pour avoir des informations. Faites donc bien attention à utiliser le {\bf clic-droit} pour déplacer une unité sur une case accessible.
		
		
		\subsection{Combat entre unités}
		
		
		
		